% Options for packages loaded elsewhere
\PassOptionsToPackage{unicode}{hyperref}
\PassOptionsToPackage{hyphens}{url}
%
\documentclass[
]{article}
\usepackage{amsmath,amssymb}
\usepackage{iftex}
\ifPDFTeX
  \usepackage[T1]{fontenc}
  \usepackage[utf8]{inputenc}
  \usepackage{textcomp} % provide euro and other symbols
\else % if luatex or xetex
  \usepackage{unicode-math} % this also loads fontspec
  \defaultfontfeatures{Scale=MatchLowercase}
  \defaultfontfeatures[\rmfamily]{Ligatures=TeX,Scale=1}
\fi
\usepackage{lmodern}
\ifPDFTeX\else
  % xetex/luatex font selection
\fi
% Use upquote if available, for straight quotes in verbatim environments
\IfFileExists{upquote.sty}{\usepackage{upquote}}{}
\IfFileExists{microtype.sty}{% use microtype if available
  \usepackage[]{microtype}
  \UseMicrotypeSet[protrusion]{basicmath} % disable protrusion for tt fonts
}{}
\makeatletter
\@ifundefined{KOMAClassName}{% if non-KOMA class
  \IfFileExists{parskip.sty}{%
    \usepackage{parskip}
  }{% else
    \setlength{\parindent}{0pt}
    \setlength{\parskip}{6pt plus 2pt minus 1pt}}
}{% if KOMA class
  \KOMAoptions{parskip=half}}
\makeatother
\usepackage{xcolor}
\usepackage[margin=1in]{geometry}
\usepackage{color}
\usepackage{fancyvrb}
\newcommand{\VerbBar}{|}
\newcommand{\VERB}{\Verb[commandchars=\\\{\}]}
\DefineVerbatimEnvironment{Highlighting}{Verbatim}{commandchars=\\\{\}}
% Add ',fontsize=\small' for more characters per line
\usepackage{framed}
\definecolor{shadecolor}{RGB}{248,248,248}
\newenvironment{Shaded}{\begin{snugshade}}{\end{snugshade}}
\newcommand{\AlertTok}[1]{\textcolor[rgb]{0.94,0.16,0.16}{#1}}
\newcommand{\AnnotationTok}[1]{\textcolor[rgb]{0.56,0.35,0.01}{\textbf{\textit{#1}}}}
\newcommand{\AttributeTok}[1]{\textcolor[rgb]{0.13,0.29,0.53}{#1}}
\newcommand{\BaseNTok}[1]{\textcolor[rgb]{0.00,0.00,0.81}{#1}}
\newcommand{\BuiltInTok}[1]{#1}
\newcommand{\CharTok}[1]{\textcolor[rgb]{0.31,0.60,0.02}{#1}}
\newcommand{\CommentTok}[1]{\textcolor[rgb]{0.56,0.35,0.01}{\textit{#1}}}
\newcommand{\CommentVarTok}[1]{\textcolor[rgb]{0.56,0.35,0.01}{\textbf{\textit{#1}}}}
\newcommand{\ConstantTok}[1]{\textcolor[rgb]{0.56,0.35,0.01}{#1}}
\newcommand{\ControlFlowTok}[1]{\textcolor[rgb]{0.13,0.29,0.53}{\textbf{#1}}}
\newcommand{\DataTypeTok}[1]{\textcolor[rgb]{0.13,0.29,0.53}{#1}}
\newcommand{\DecValTok}[1]{\textcolor[rgb]{0.00,0.00,0.81}{#1}}
\newcommand{\DocumentationTok}[1]{\textcolor[rgb]{0.56,0.35,0.01}{\textbf{\textit{#1}}}}
\newcommand{\ErrorTok}[1]{\textcolor[rgb]{0.64,0.00,0.00}{\textbf{#1}}}
\newcommand{\ExtensionTok}[1]{#1}
\newcommand{\FloatTok}[1]{\textcolor[rgb]{0.00,0.00,0.81}{#1}}
\newcommand{\FunctionTok}[1]{\textcolor[rgb]{0.13,0.29,0.53}{\textbf{#1}}}
\newcommand{\ImportTok}[1]{#1}
\newcommand{\InformationTok}[1]{\textcolor[rgb]{0.56,0.35,0.01}{\textbf{\textit{#1}}}}
\newcommand{\KeywordTok}[1]{\textcolor[rgb]{0.13,0.29,0.53}{\textbf{#1}}}
\newcommand{\NormalTok}[1]{#1}
\newcommand{\OperatorTok}[1]{\textcolor[rgb]{0.81,0.36,0.00}{\textbf{#1}}}
\newcommand{\OtherTok}[1]{\textcolor[rgb]{0.56,0.35,0.01}{#1}}
\newcommand{\PreprocessorTok}[1]{\textcolor[rgb]{0.56,0.35,0.01}{\textit{#1}}}
\newcommand{\RegionMarkerTok}[1]{#1}
\newcommand{\SpecialCharTok}[1]{\textcolor[rgb]{0.81,0.36,0.00}{\textbf{#1}}}
\newcommand{\SpecialStringTok}[1]{\textcolor[rgb]{0.31,0.60,0.02}{#1}}
\newcommand{\StringTok}[1]{\textcolor[rgb]{0.31,0.60,0.02}{#1}}
\newcommand{\VariableTok}[1]{\textcolor[rgb]{0.00,0.00,0.00}{#1}}
\newcommand{\VerbatimStringTok}[1]{\textcolor[rgb]{0.31,0.60,0.02}{#1}}
\newcommand{\WarningTok}[1]{\textcolor[rgb]{0.56,0.35,0.01}{\textbf{\textit{#1}}}}
\usepackage{longtable,booktabs,array}
\usepackage{calc} % for calculating minipage widths
% Correct order of tables after \paragraph or \subparagraph
\usepackage{etoolbox}
\makeatletter
\patchcmd\longtable{\par}{\if@noskipsec\mbox{}\fi\par}{}{}
\makeatother
% Allow footnotes in longtable head/foot
\IfFileExists{footnotehyper.sty}{\usepackage{footnotehyper}}{\usepackage{footnote}}
\makesavenoteenv{longtable}
\usepackage{graphicx}
\makeatletter
\def\maxwidth{\ifdim\Gin@nat@width>\linewidth\linewidth\else\Gin@nat@width\fi}
\def\maxheight{\ifdim\Gin@nat@height>\textheight\textheight\else\Gin@nat@height\fi}
\makeatother
% Scale images if necessary, so that they will not overflow the page
% margins by default, and it is still possible to overwrite the defaults
% using explicit options in \includegraphics[width, height, ...]{}
\setkeys{Gin}{width=\maxwidth,height=\maxheight,keepaspectratio}
% Set default figure placement to htbp
\makeatletter
\def\fps@figure{htbp}
\makeatother
\setlength{\emergencystretch}{3em} % prevent overfull lines
\providecommand{\tightlist}{%
  \setlength{\itemsep}{0pt}\setlength{\parskip}{0pt}}
\setcounter{secnumdepth}{5}
\newlength{\cslhangindent}
\setlength{\cslhangindent}{1.5em}
\newlength{\csllabelwidth}
\setlength{\csllabelwidth}{3em}
\newlength{\cslentryspacingunit} % times entry-spacing
\setlength{\cslentryspacingunit}{\parskip}
\newenvironment{CSLReferences}[2] % #1 hanging-ident, #2 entry spacing
 {% don't indent paragraphs
  \setlength{\parindent}{0pt}
  % turn on hanging indent if param 1 is 1
  \ifodd #1
  \let\oldpar\par
  \def\par{\hangindent=\cslhangindent\oldpar}
  \fi
  % set entry spacing
  \setlength{\parskip}{#2\cslentryspacingunit}
 }%
 {}
\usepackage{calc}
\newcommand{\CSLBlock}[1]{#1\hfill\break}
\newcommand{\CSLLeftMargin}[1]{\parbox[t]{\csllabelwidth}{#1}}
\newcommand{\CSLRightInline}[1]{\parbox[t]{\linewidth - \csllabelwidth}{#1}\break}
\newcommand{\CSLIndent}[1]{\hspace{\cslhangindent}#1}
\ifLuaTeX
  \usepackage{selnolig}  % disable illegal ligatures
\fi
\IfFileExists{bookmark.sty}{\usepackage{bookmark}}{\usepackage{hyperref}}
\IfFileExists{xurl.sty}{\usepackage{xurl}}{} % add URL line breaks if available
\urlstyle{same}
\hypersetup{
  pdftitle={Hubungan Antara Kebersihan Lingkungan dan Kesehatan dengan Korelasi Pencemaran Air dan DBD},
  pdfauthor={23611094\_Nailul lMuna},
  hidelinks,
  pdfcreator={LaTeX via pandoc}}

\title{Hubungan Antara Kebersihan Lingkungan dan Kesehatan dengan
Korelasi Pencemaran Air dan DBD}
\author{23611094\_Nailul lMuna}
\date{2025-07-15}

\begin{document}
\maketitle

\includegraphics{https://iik.ac.id/blog/wp-content/uploads/2023/01/kesehatan-lingkungan.jpeg}

Kesehatan masyarakat sangat dipengaruhi oleh faktor lingkungan, termasuk
kebersihan dan tingkat pencemaran yang ada di sekitar kita. Banyak
penelitian menunjukkan bahwa kebersihan lingkungan berhubungan erat
dengan kesehatan, di mana pencemaran lingkungan dapat meningkatkan
risiko berbagai penyakit. Artikel ini akan membahas hubungan antara
kebersihan lingkungan, khususnya pencemaran air, dengan kesehatan,
terutama Demam Berdarah Dengue (DBD). Dalam analisis ini, kita juga akan
mengintegrasikan penelitian-penelitian terkait faktor risiko infeksi,
seperti pada Buruli ulcer, plague di Madagascar, dan manajemen infeksi
pada pasien dengan kondisi tertentu. Penelitian mengenai faktor risiko
untuk \emph{Buruli ulcer} di rumah sakit misi rujukan mengungkapkan
bagaimana kondisi lingkungan berperan penting dalam penyebaran infeksi
kulit ini. Buruli ulcer, yang merupakan infeksi kulit yang terkait erat
dengan lingkungan, menunjukkan prevalensi yang lebih tinggi di
daerah-daerah dengan sanitasi yang buruk dan akses terbatas ke layanan
kesehatan. Temuan ini memperkuat pemahaman bahwa kebersihan lingkungan
memiliki dampak signifikan terhadap kesehatan masyarakat, khususnya
dalam pencegahan penyakit infeksi. (Kamble and Deshmukh 2021) Studi
tentang patogen baru yang menyoroti wabah \emph{plague} di Madagascar
memberikan perspektif lebih luas tentang hubungan antara lingkungan dan
penyebaran penyakit. Plague yang disebabkan oleh bakteri Yersinia pestis
ini menunjukkan bagaimana faktor-faktor seperti sanitasi yang buruk dan
kondisi lingkungan yang mendukung pertumbuhan vektor penyakit dapat
memfasilitasi penyebaran patogen berbahaya. Kasus di Madagascar ini
menjadi contoh nyata bagaimana kebersihan lingkungan yang tidak memadai
dapat berkontribusi pada munculnya wabah penyakit yang serius.
(Rakotoarivony and al. 2019) Penelitian tentang insidensi dan manajemen
infeksi pada pasien dengan penyakit tertentu juga menekankan peran
krusial lingkungan dalam kesehatan. Studi ini menunjukkan bahwa pasien
dengan kondisi medis tertentu menjadi lebih rentan terhadap infeksi
tambahan ketika berada di lingkungan yang terkontaminasi dengan sanitasi
yang buruk. Hal ini menggarisbawahi pentingnya pengelolaan lingkungan
yang baik tidak hanya untuk pencegahan penyakit primer, tetapi juga
untuk mencegah komplikasi infeksi sekunder yang dapat memperburuk
kondisi kesehatan pasien secara keseluruhan. (Kumari and al. 2020)

\hypertarget{korelasi-antara-kebersihan-lingkungan-dan-kesehatan}{%
\subsection{Korelasi antara Kebersihan Lingkungan dan
Kesehatan}\label{korelasi-antara-kebersihan-lingkungan-dan-kesehatan}}

\textbf{Korelasi} adalah ukuran statistik yang menunjukkan hubungan atau
asosiasi antara dua variabel. Dalam konteks kebersihan lingkungan dan
kesehatan, korelasi digunakan untuk mengukur sejauh mana perubahan dalam
kebersihan lingkungan (misalnya, peningkatan pencemaran udara, tanah,
atau air) berhubungan dengan perubahan dalam tingkat kesehatan
(misalnya, peningkatan kasus penyakit). Koefisien korelasi ini dapat
bernilai positif, negatif, atau tidak ada korelasi sama sekali. Korelasi
positif menunjukkan bahwa ketika satu variabel meningkat, variabel
lainnya juga meningkat, sedangkan korelasi negatif menunjukkan bahwa
ketika satu variabel meningkat, variabel lainnya menurun. Jika tidak ada
korelasi, maka perubahan pada satu variabel tidak mempengaruhi variabel
lainnya.

\hypertarget{rumus-korelasi-pearson}{%
\subsection{Rumus Korelasi Pearson}\label{rumus-korelasi-pearson}}

Koefisien korelasi yang paling sering digunakan adalah koefisien
korelasi Pearson, yang mengukur kekuatan dan arah hubungan linear antara
dua variabel. Rumusnya adalah: \[
r = \frac{n \left( \sum xy \right) - \left( \sum x \right) \left( \sum y \right)}{\sqrt{ \left[ n \sum x^2 - \left( \sum x \right)^2 \right] \left[ n \sum y^2 - \left( \sum y \right)^2 \right] }}
\]

Di mana:

\begin{itemize}
\tightlist
\item
  r adalah koefisien korelasi Pearson,
\item
  x dan y adalah dua variabel yang ingin diukur korelasinya (misalnya,
  tingkat pencemaran air dan jumlah kasus DBD),
\item
  n adalah jumlah data yang digunakan.
\end{itemize}

Interpretasi Nilai Korelasi Nilai koefisien korelasi r berkisar antara
-1 hingga 1:

\begin{itemize}
\tightlist
\item
  r=1: Hubungan positif sempurna (satu variabel naik, variabel lainnya
  juga naik)
\item
  r=−1: Hubungan negatif sempurna (satu variabel naik, yang lainnya
  turun).
\item
  r=0: Tidak ada hubungan linear antara dua variabel.
\end{itemize}

\hypertarget{contoh-korelasi-pada-pencemaran-air-dan-dbd}{%
\subsection{Contoh Korelasi pada Pencemaran Air dan
DBD}\label{contoh-korelasi-pada-pencemaran-air-dan-dbd}}

Dalam analisis ini, kita mengukur hubungan antara pencemaran air dan
kasus DBD di beberapa provinsi. Korelasi yang ditemukan menunjukkan
adanya hubungan positif yang sangat kuat antara pencemaran air dan
peningkatan jumlah kasus DBD. Artinya, semakin tinggi tingkat pencemaran
air, semakin banyak kasus DBD yang tercatat.

\hypertarget{data-dan-analisis}{%
\subsection{Data dan Analisis}\label{data-dan-analisis}}

Berikut adalah bagian dari data yang digunakan untuk analisis:

\begin{longtable}[]{@{}lll@{}}
\toprule\noalign{}
Provinsi & DBD & Pencemaran Air \\
\midrule\noalign{}
\endhead
\bottomrule\noalign{}
\endlastfoot
Aceh & 1533 & 729 \\
Sumatera Utara & 5623 & 1205 \\
Sumatera Barat & 2203 & 319 \\
Riau & 918 & 454 \\
Jambi & 720 & 614 \\
\end{longtable}

\begin{Shaded}
\begin{Highlighting}[]
\CommentTok{\# Memuat library yang dibutuhkan}
\FunctionTok{library}\NormalTok{(seeCorrelation)}
\FunctionTok{library}\NormalTok{(readxl)}
\end{Highlighting}
\end{Shaded}

\begin{verbatim}
## Warning: package 'readxl' was built under R version 4.3.3
\end{verbatim}

\begin{Shaded}
\begin{Highlighting}[]
\CommentTok{\# Membaca data dari file Excel}
\NormalTok{kesehatan }\OtherTok{\textless{}{-}} \FunctionTok{read\_excel}\NormalTok{(}\StringTok{"C://Una\textquotesingle{}s\_Files//SEMESTER\_4//PRAK\_KS//eval4\_1//data\_eval4.xlsx"}\NormalTok{)}

\CommentTok{\# Melakukan analisis korelasi antara DBD dan Pencemaran Air}
\FunctionTok{see\_correlation}\NormalTok{(kesehatan, }\StringTok{"DBD"}\NormalTok{, }\StringTok{"Pencemaran\_air"}\NormalTok{, }\AttributeTok{metode =} \StringTok{"cor"}\NormalTok{)}
\end{Highlighting}
\end{Shaded}

\begin{verbatim}
## [1] "Nilai korelasi antara DBD dan Pencemaran_air adalah 0.818047773981473"
\end{verbatim}

\includegraphics{correlation_artikel_files/figure-latex/unnamed-chunk-1-1.pdf}

\begin{verbatim}
## [1] 0.8180478
\end{verbatim}

Nilai korelasi antara DBD dan Pencemaran Air sebesar 0.818 menunjukkan
hubungan positif yang sangat kuat. Ini berarti bahwa semakin tinggi
tingkat pencemaran air, semakin banyak kasus DBD yang tercatat di
berbagai provinsi. Hubungan ini mengindikasikan bahwa pencemaran air
berperan signifikan dalam meningkatkan risiko penyakit seperti DBD, yang
seringkali dipengaruhi oleh kondisi sanitasi dan kebersihan lingkungan.

\hypertarget{kesimpulan}{%
\subsection{Kesimpulan}\label{kesimpulan}}

\textbf{Berdasarkan analisis korelasi yang dilakukan}, dapat disimpulkan
bahwa kebersihan lingkungan, khususnya kualitas air, sangat memengaruhi
kesehatan masyarakat. Pencemaran air yang tinggi dapat meningkatkan
angka kejadian penyakit seperti DBD. Oleh karena itu, penting untuk
meningkatkan kualitas air dan mengurangi tingkat polusi di berbagai
daerah agar dapat menurunkan angka penyakit seperti DBD. Selain itu,
hasil ini juga mendukung temuan dalam penelitian terkait
penyakit-penyakit lainnya, seperti Buruli ulcer dan plague, yang
menunjukkan bahwa lingkungan yang tidak bersih dapat meningkatkan risiko
berbagai infeksi dan penyakit. Melalui upaya pengendalian pencemaran air
dan perbaikan kebersihan lingkungan, diharapkan dapat mengurangi dampak
kesehatan yang ditimbulkan oleh pencemaran dan meningkatkan kualitas
hidup masyarakat.

\hypertarget{referensi}{%
\section*{Referensi}\label{referensi}}
\addcontentsline{toc}{section}{Referensi}

\hypertarget{refs}{}
\begin{CSLReferences}{1}{0}
\leavevmode\vadjust pre{\hypertarget{ref-kamble2021risk}{}}%
Kamble, S. S., and P. B. Deshmukh. 2021. {``Risk Factors for Buruli
Ulcer in a Referral Mission Hospital.''} \emph{SciSpace}.

\leavevmode\vadjust pre{\hypertarget{ref-kumari2020incidence}{}}%
Kumari, A., and et al. 2020. {``Incidence and Management of Infections
in Patients with Chronic Conditions.''} \emph{SciSpace}.

\leavevmode\vadjust pre{\hypertarget{ref-rakotoarivony2019emerging}{}}%
Rakotoarivony, I. M., and et al. 2019. {``Emerging Pathogens: The Plague
in Madagascar.''} \emph{SciSpace}.

\end{CSLReferences}

\end{document}
